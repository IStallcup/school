\documentclass[10pt,draftclsnofoot,onecolumn,compsoc]{IEEEtran}

\usepackage{hyperref}
\usepackage{pdfpages}
\usepackage{float}
\usepackage{listings}

\usepackage{color}
\definecolor{blue}{rgb}{0.13,0.13,1}
\definecolor{green}{rgb}{0,0.5,0}
\definecolor{red}{rgb}{0.9,0,0}


\lstdefinelanguage{clangstyle}%
{	keywordstyle=\color{blue},
	sensitive=true,
	basicstyle=\ttfamily,
	breaklines=true,
	xleftmargin=\parindent,
	aboveskip=\bigskipamount,
	tabsize=4,
	morecomment=[l][\color{green}]{//},
	showstringspaces=false,
	literate={`}{\`}1,
	stringstyle=\color{red},
}


\author{Isaac Stallcup}

\begin{document}
	
	\begin{titlepage}
		
		\begin{center}

		\fontsize{20pt}{20pt}\selectfont
				
		\vspace*{3.5cm}
	
		\textbf{Writing Assignment 1}\\
		
		\vspace{0.5cm}
		
		\fontsize{16pt}{16pt}\selectfont
		
		Isaac Stallcup \\
		CS 444\\
		Spring 2017\\
		

		\end{center}
	
	\end{titlepage}
	
	\section{Processes}
	
	\subsection{Windows}
	
	In Windows, processes begin with one thread, called the primary thread. However, each thread within a process can create additional threads \cite{msdnproc}. Threads in Windows share the virtual address space and system resource availability of the process they belong to. 
	
	Processes in Windows use virtual memory addresses when reading or writing to memory. \cite{msdnproc} Virtual memory addresses are placeholders, and when read or write operations with virtual memory addresses are called the virtual memory address is assigned to a physical memory address. \cite{msdnvirt} Processes also have handles to system resources. Handles are tools used to interact with system resource objects, like files or threads. 
	
	Within a process, each of its threads also have priorities associated with them, valued from 0 to 31. \cite{msdnsched} The scheduling utility in Windows then assigns time slices to be scheduled in round-robin fashion, starting with highest priority threads then moving to the lowest priority threads \cite{msdnproc}.
	
	
	\subsection{FreeBSD}

	FreeBSD has two methods of implementing threading. The first, 1:1 threading, implements each thread as an individual process. The second, 1:N threading, groups all threads as a single process and runs it that way. Both have benefits and detriments. \cite{bsdsched}
	
	1:1 scheduling implements larger threads than 1:N scheduling, and in 1:1 scheduling the scheduling itself cannot be altered by the user where as in 1:N scheduling the user can modify the scheduling. However, 1:1 threading can utilize multiple CPUs, and 1:N threading cannot. \cite{bsdsched}
	
	FreeBSD implements scheduling in two ways; that is, there are two schedulers that come with FreeBSD. They are M:N (hybrid) threading and 1:1 threading. \cite{bsdsched} 
	

	\subsection{Comparisons to Linux}
	
	Windows and Linux both use virtual memory \cite{kernel} \cite{msdnvirt}. 

\bibliographystyle{IEEEtran}	
\bibliography{biblio}
\end{document}